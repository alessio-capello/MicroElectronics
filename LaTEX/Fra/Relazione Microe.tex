% !TEX TS-program = pdflatex
% !TEX encoding = UTF-8 Unicode

% This is a simple template for a LaTeX document using the "article" class.
% See "book", "report", "letter" for other types of document.

\documentclass[11pt]{article} % use larger type; default would be 10pt

\usepackage[utf8]{inputenc} % set input encoding (not needed with XeLaTeX)

%%% Examples of Article customizations
% These packages are optional, depending whether you want the features they provide.
% See the LaTeX Companion or other references for full information.

%%% PAGE DIMENSIONS
\usepackage{geometry} % to change the page dimensions
\geometry{a4paper} % or letterpaper (US) or a5paper or....
% \geometry{margin=2in} % for example, change the margins to 2 inches all round
% \geometry{landscape} % set up the page for landscape
%   read geometry.pdf for detailed page layout information

\usepackage{graphicx} % support the \includegraphics command and options

% \usepackage[parfill]{parskip} % Activate to begin paragraphs with an empty line rather than an indent

%%% PACKAGES
\usepackage{booktabs} % for much better looking tables
\usepackage{array} % for better arrays (eg matrices) in maths
\usepackage{paralist} % very flexible & customisable lists (eg. enumerate/itemize, etc.)
\usepackage{verbatim} % adds environment for commenting out blocks of text & for better verbatim
\usepackage{subfig} % make it possible to include more than one captioned figure/table in a single float
% These packages are all incorporated in the memoir class to one degree or another...

%%% HEADERS & FOOTERS
\usepackage{fancyhdr} % This should be set AFTER setting up the page geometry
\pagestyle{fancy} % options: empty , plain , fancy
\renewcommand{\headrulewidth}{0pt} % customise the layout...
\lhead{}\chead{}\rhead{}
\lfoot{}\cfoot{\thepage}\rfoot{}

%%% SECTION TITLE APPEARANCE
\usepackage{sectsty}
\allsectionsfont{\sffamily\mdseries\upshape} % (See the fntguide.pdf for font help)
% (This matches ConTeXt defaults)

%font sezioni ecc
\usepackage{titlesec}

\titleformat{\section} {\normalfont\large\bfseries\centering}{\thesection}{1em}{}
\titleformat{\subsection} {\normalfont\large}{\thesubsection}{1em}{}
\titleformat{\subsubsection} {\normalfont\normalsize}{\thesubsubsection}{1em}{}
\titleformat{\paragraph}[runin] {\normalfont\normalsize}{\theparagraph}{1em}{}
\titleformat{\subparagraph}[runin] {\normalfont\normalsize}{\thesubparagraph}{1em}{}

%%% ToC (table of contents) APPEARANCE
\usepackage[nottoc,notlof,notlot]{tocbibind} % Put the bibliography in the ToC
\usepackage[titles,subfigure]{tocloft} % Alter the style of the Table of Contents
\renewcommand{\cftsecfont}{\rmfamily\mdseries\upshape}
\renewcommand{\cftsecpagefont}{\rmfamily\mdseries\upshape} % No bold!

%%% END Article customizations

%%% The "real" document content comes below...

\title{TSPC 4-bit Full Adder}
\author{Francesco Morgillo\\3810105}
%\date{} % Activate to display a given date or no date (if empty),
         % otherwise the current date is printed 

\begin{document}
\maketitle

\section{Obiettivi del progetto B}
Si presenta il progetto di design di un full-adder a 4 bit con tecnologia MOS 0.12 $\mu$ m e logica TSPC (true single phase clock). 
Il Full Adder presenta quattro ingressi: due word a 4 bit di cui eseguire la somma ($a_{0},b_{0}$), un $CARRY IN$ ($ c_{0}$) e un ingresso dedicato al clock ($\phi$).
Le uscite del circuito sono i segnali di $SUM$ e $CARRY$.
\subsection{Specifiche del Progetto}
Per la realizzazione del circuito sono richieste l'utilizzo di tecnologia MOS a 0.12 $\mu$ m, un'alimentazione a $1.2V$, frequenza di lavoro $2GHz$,  carico capacitivo di $100fF$ e una dimensione della cella con altezza massima di 68$\lambda$.

\section{Logica TSPC}
La logica TSPC prevede l'utilizzo di un blocco di logica statica e di un segnale di clock che temporizza uno stadio latch. Si hanno due fasi differenti: una di precarica (quando $\phi=0$) ed una di valutazione (quando $\phi=1$). Il segnale di uscita va considerato in fase di valutazione.
\clearpage
\section{Dimensionamento}
Sono state considerate le specifiche relative alla frequenza di funzionamento e al carico per ottenere le dimensioni dei transistor, nei vari stadi.  
E' necessario innanzitutto calcolare le dimensioni di $W_{n}$ dello stadio finale. Calcolando il carico equivalente di ogni stadio, si ottiene a ritroso il valore di $W_{n}$ di ogni stadio.
Le dimensioni $W_{n}$ si ottengono con la formula seguente:

\begin{equation} 
\frac{W_{n}}{L}=\frac{2CV_{dd}}{\
 \mu_{n}C_{ox}(V_{gs}-V_{th})^2}
\end{equation}
Dove:\mbox{}\\

\begin{tabular}{|ll|}
\hline
$V_{gs}$ & $1.20V$ \\
$C$ & $100fF$ \\
$\tau$ & 250ps\\
$\mu_{n}$ & $0.06$  $\frac{fF}{m^2}$  \\
$C_{ox}$ & $0.01725$ $\frac{m^2}{Vs}$  \\
$V_{gs} - V_{th}$ & $0.8V$ \\
\hline
\end{tabular}
\mbox{}\\
\\
Partendo dal terzo stadio si applica la formula e considerando il vincolo di tecnologia $L= 120 nm$ di ottiene un rapporto $\frac {W_{n}}{L}= 1.4486$.
Si approssima all'intero successivo visto che larghezza  $W_{n}$ non può che essere un multiplo intero di L :   $W_{n}=2L= 240nm$.
Per un transistor $p$ si dovrà triplicare questo valore per via della minor mobilità degli elettroni nel substrato $P$, quindi $W_{p}=3W_{n}= 720nm$.

Per quanto riguarda il secondo stadio è necessario considerare la capacità equivalente che esso vede sul terzo stadio.
La capacità equivalente viene calcolata come segue :
\begin{equation}
C_{G}= C_{ox}W_{n}L+C_{ox}W_{p}L= 2.93fF
\end{equation}
Il valore del carico da pilotare è abbastanza piccolo per poter utilizzare dei transistor a dimensione minima ($W_{n}=120 nm$, $W_{p}=3W_{n}= 360nm$)
Lo stesso ragionamento porta a scegliere dei transistor a dimensione minima anche nel primo stadio. 

\subsection{}

More text.

\end{document}
